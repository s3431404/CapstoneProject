% RMIT University School of CS&IT
% Minor thesis template
% S.M.M. (Saied) Tahaghoghi, 2004
\documentclass[11pt,twoside]{report}
\usepackage{a4wide,caption,epsfig,fancyheadings,natbib,url,multirow,makecell}

% Place the correct values here
%Set to the original submission date when submitted amended thesis
\newcommand{\SubmissionDate}{\today}
\newcommand{\student}{Bin Lu}
\newcommand{\supervisor}{Dr Jenny Zhang, Dr Amanda Kimpton, Dr Daryl D'Souza}
\newcommand{\topic}{User-driven topic modeling of patient opinion data}
\newcommand{\school}{School of Computer Science and Information Technology}
\newcommand{\program}{Masters of Computer Science}
\newcommand{\institution}{Royal Melbourne Institute of Technology}

% Use the remark command to highlight text for discussion
\newcommand{\remark}[1]{{\bf \em [\marginpar{$\Leftarrow$}#1]}}

\renewcommand{\leftmark}{\student}
\renewcommand{\rightmark}{\topic}
%\setlength{\headrulewidth}{0pt}
\setlength{\parindent}{0pt}
\setlength{\parskip}{1.5ex plus 0.3ex}

% This is the line spacing - set to 2 for draft submission to
% supervisor, 1.3 for the final submission
\renewcommand{\baselinestretch}{1.3}

\renewcommand{\captionfont}{\it}
\raggedbottom

%For Natbib Author, year citation format
% - the opening bracket symbol, default = (
% - the closing bracket symbol, default = )
% - the punctuation between multiple citations, default = ;
% - the letter n for numerical style, or s for numerical superscript
%   style, any other letter for author year, default = author year;
% - the punctuation that comes between the author names and the year
% - the punctuation that comes between years or numbers when common author lists are suppressed (default = ,);
\bibpunct{[}{]}{;}{a}{,}{;}


\begin{document}

%%%%%%%%%%%%%%%%%%%%%%%%%%%%%%%%%%%%%%%%%%%%%%%%%%%%%%%%%%%%%%%%%%%%%%
\title{{\Large\bf \topic}}
\author{
A minor thesis submitted in partial fulfilment of the requirements for the degree of
\\\program\\*[10mm]
%\epsfig{figure=Figs/rmit-coa.epsf,width=5cm}
\\\student
\\\school
\\Science, Engineering, and Technology Portfolio,
\\\institution
\\Melbourne, Victoria, Australia
}
\maketitle
\thispagestyle{empty}


%%%%%%%%%%%%%%%%%%%%%%%%%%%%%%%%%%%%%%%%%%%%%%%%%%%%%%%%%%%%%%%%%%%%%%
\chapter*{Declaration}

This thesis contains work that has not been submitted previously, in
whole or in part, for any other academic award and is solely my
original research, except where acknowledged.

This work has been carried out since March 2014, under the
supervision of {\supervisor}.

\paragraph{}
\vspace{5cm}\noindent \\\student \\
\school\\
\institution\\
\SubmissionDate

\pagenumbering{roman}

%%%%%%%%%%%%%%%%%%%%%%%%%%%%%%%%%%%%%%%%%%%%%%%%%%%%%%%%%%%%%%%%%%%%%%
\chapter*{Acknowledgements}

I would like to take this opportunity to convey my special thanks to my supervisors Dr Jenny Zhang, Dr Amanda Kimpton and Dr Daryl D'Souza for their incredible support during the past year. I was so lucky to have three best supervisors from school which gives me opportunity to extend my knowledge in different directions, and my work progress is constantly reviewed from 3 different perspectives. 


\tableofcontents
\listoffigures
\listoftables

\pagenumbering{arabic}
%%%%%%%%%%%%%%%%%%%%%%%%%%%%%%%%%%%%%%%%%%%%%%%%%%%%%%%%%%%%%%%%%%%%%%
\chapter*{Abstract}
Topic models are increasingly being deployed as a powerful technique to discover useful structure or topics in otherwise unstructured, large text collections. Such collections are widely available in the form of web-based forums for consumers to express feedback (opinions, reviews, general commentary) about products and services, thereby allowing providers to improve products and services. Such improvements are dependent on effective decision-making from the data available in these forums. In turn, such vast quantities of (typically content-specific) data call for techniques to accurately classify and summarise forum posts. Topic models have emerged as a useful way of automatically finding useful structure in such large, collections of free-form text data, to aid providers and domain experts in arriving at effective decisions (about products and services). While many studies have addressed this problem, none, to our knowledge, has applied topic modeling to healthcare. In this thesis we develop a topic model for a web-based forum for patient feedback (patientopinion.org.au) about general healthcare issues. Our model provides improvements of the baseline model generated by a popular topic modeling tool (Mallett) through the use of user tag data available in the posts within patientopinion.org.au.  

%%%%%%%%%%%%%%%%%%%%%%%%%%%%%%%%%%%%%%%%%%%%%%%%%%%%%%%%%%%%%%%%%%%%%%
\chapter{Introduction}
Publicly available web-based
forums for consumer feedback about products and services 
provide valuable information for decision making for both consumers and
service providers (of products and services) alike. 
With the availability of forums such as blogs and social networks, 
it has never been easier to freely, critically and anonymously express 
feedback in the form of opinions and general comments about products and 
services.
Analysing the feedback poses challenges, in part because of the 
vast quantities of user posts, but also because the responses from general
consumers is typically expressed 
in free form text. Both providers and consumers use such forums in a variety 
of ways. Providers use the data to effect improvements in products and services.
Consumers rely on other consumers' feedback to decide on acquisition of
products and services, or to validate their own opinions be it about complaints
or the comments about improved services. A provider might wish to identify 
a service (among many of its services) that is causing customers to withdraw
their patronage. A consumer might wish to locate a provider outlet with the 
friendliest customer service.
These decisions cannot be made without manually scanning large volumes of 
postings in the forum. If forums are regarded as collections of large, free 
form text documents (a post by a consumer may be regarded as a single document)
then effective decision making from the feedback in these collections is
impossible without 
systematically analysing, summarizing and subsequently
organising the documents to qualitatively enhance the usability 
of the available data. 

And many previous studies focus on analysing product reviews. The focus of 
this study is to investigate a model that on suite reviews and more specifically, 
reviews related to health systems. Previous studies have shown that effective 
regulation of healthcare professions is increasingly recognized as pivotal to 
improvements in healthcare quality (\cite{ref6}). 
Moreover the effectiveness of regulatory authorities is affected by 
insufficient resources and inadequate information received (\cite{ref5}). 

The official authority called the Australian Health Practitioner Regulation 
Agency (AHPRA) is responsible to monitor performance and conduct of health 
practitioners across 14 health professions. Since 2010 laws in all Australian 
states and territories require health practitioners to report 
all ``notifiable conduct" that comes to their attention to AHPRA. 
The law targets all registered health practitioners in Australia 
which includes doctors, nurses, dentists and practitioners from 
11 allied health professions. The obligation to report notifiable 
conduct includes an obligation on behalf of employers, 
education providers and health practitioners. A report can be made 
if there is a reasonable belief that the behaviour or performance is notifiable. 

Apart from the role of AHPRA to facilitate the regulation of 14 
healthcare professions, there are other forums to address 
poor performing healthcare practitioners. These include publicly 
available stories, feedbacks and opinions from patient which can address 
similar issues faced by AHPRA.  

In this thesis we
focus on topic modeling as a way to summarise documents in such web-based 
forums or collections. Specifically, we apply topic modeling to a web-based
forum that deals with patient opinion (www.patientopinion.org.au). 
The following quote summaries the objectives of the 
forum\footnote{www.patientopinion.org.au/info/about}:

\begin{quote}Patient Opinion was founded in the UK in 2005 and since then has grown 
to be the UK's leading independent non-profit feedback platform for health 
services. Patient Opinion Australia (POA) was established in 2012 and, 
similar to its UK counterpart, is registered as an independent not-for-profit 
charitable institution. Patient Opinion is about honest and meaningful 
conversations between patients and health services. We believe that your 
story can help make health services better.
\end{quote}

A \emph{topic} is a set of highly probable words or terms 
discovered from the collection
of documents that
are used to categorise a subset
of the documents in the collection. 
Topic modeling automatically discovers a set of
identifying topics that allow for effective organisation of the collection
of documents. 

Topic modeling has proven a very powerful tool of analysing large quantity of data, on the other hand publicly available healthcare opinions, reviews have been overlooked in previous topic modeling studies.

For this project we use Latent Dirichlet Allocation (LDA) as a base model which has been used in many other studies as base model and has proven to be effective. For example, Multi-Domain Prior Knowledge - LDA (MDK-LDA) proposed by Chen (\cite{ref24}),  extends the Latent Dirichlet Allocation (\cite{ref25}). MDK-LDA introduces a new latent variable s in LDA to model s-sets which is a domain expert pre-defined category or group. Each document is an admixture of latent topics while each topic is a probability distribution over s-sets. Another approach is Aspect-Based Summarization (\cite{ref11}). It is usually composed of three main tasks: aspect identification, sentiment classification, and aspect rating. This model is used to analyse product review and designed to effectively retrieve features and sentiment for products.

Due to the unique characteristic of the data from ``Patient Opinion", the existing 
algorithm has been improved with user input data. LDA has proven to be a 
very effective  model and has been used as a based model in many topic modeling 
studies. We chose LDA as our base model, and incorporated unique features of ``Patient Opinion", specifically the sections of the website includes ``What is Good" and 
``What could be improved". These two sections are completed by users 
for submission of the story. The template for submission of stories is 
provided by the website. Generally ``What is Good" and 
``What could be improved" are the main topic or features users want to give 
feedback about in their stories. And we assume user labelled
story 100\% accurate.

The research questions we are going to answer are:
\begin{itemize}
\item Does user-driven input improve topic modeling?
\end{itemize}

The project has made the following contribution to the field of topic modeling by using the LDA as the base framework:
Introduction of the user specified keywords, and reduction in the number of terms for each topic whilst retaining the quality of the topic. The remaining terms are ranked for each topic by inverse document frequency score which is a common metrics in information retrieval area to measure how much information a word provides, that is, whether the term is common or rare across all document. 

%%%%%%%%%%%%%%%%%%%%%%%%%%%%%%%%%%%%%%%%%%%%%%%%%%%%%%%%%%%%%%%%%%%%%%
\chapter{Related Work}

% ~~~~~~~~~~~~~~~~~~~~~~~~~~
Latent Dirichlet Allocation also known as LDA or discrete PCA is a Bayesian graphical model for text document collections represented by bags-of-words (\cite{ref26}, \cite{ref25}, \cite{ref30}, \cite{ref31}). The basic idea of LDA is to treat each document in a collection as a vector of word count. Each document is represented as a probability distribution over a number of topics, while each topic is represented as a probability distribution over a number of words. The model allows sets of observations to be explained by unobserved groups that explain why some parts of the data are similar. Generally, only a small number of words have high likelihood in each topic and each document only presents certain number of topics. Following is the equation of collapsed Gibbs sampling:
\begin{equation}
p(z_{id}=t\mid x_{id}=w,Z^{\neg id}) \alpha
\end{equation}
\begin{equation}
\frac{N_{wt}^{\neg id} + \beta}{\sum_{w}N_{wt}^{\neg id} + W\beta} \frac{N_{wt}^{\neg id} + \alpha}{\sum_{w}N_{wt}^{\neg id} + T\alpha}
\end{equation}

where $z_{id}=t$ assigns topic t with $i^{th}$ word in document d, and word w currently observed indicated by $x_{id}=w$.
$Z^{\neg id}$ is the vector of all topic assignments not including the current word. $N_{wt}$ represent integer count arrays, and $\alpha$ is the parameter of the Dirichlet prior on the per-document topic distributions, $\beta$ is the per-topic word distribution.

% ~~~~~~~~~~~~~~~~~~~~~~~~~~
\section{Multi-Domain Prior Knowledge in Topic Modeling}
As mentioned before, LDA is a powerful topic modeling framework, however recent 
studies found that these unsupervised models may not produce topics that conform to 
the user's existing knowledge(\cite{ref24}). Chen et al (\cite{ref24}) proposed a 
novel knowledge-based model, called Multi-Domain Prior Knowledge -LDA (MDK-LDA), which is capable of using prior knowledge 
from multiple domains to help topic modeling in the new domain. 
A new latent variable ``s" is added to model the s-set and this s-set is defined by domain expert, each document represent 
admixture of latent topics while each topic is a probability distribution over s-set. 
MDK-LDA uses s-set to distinguish topics in multiple senses. 
For example the word light can be represented by two s-set: S1 \{light, heavy, weight\} and S2 \{light, bright, luminance\}, if light co-occurs with bright or 
luminance it will be assigned to S2.
The words in each topic are ranked by per-topic word distribution $\pi _{t}(w) = \sum_{s=1}^s (\psi_{t}(s)\cdot \eta_{t,s}(w) )$ where \psi_t{s} is a per topic distribution over s-sets and \eta_{t,s}(w) is per topic, per s-set distribution over word. To evaluate the discovered topics, two human domain experts are employed, top 20 words from each topic are selected based one previous ranking. Each word in a topic is considered correct if both experts agree, otherwise the word is labelled as incorrect. A natural way to evaluated these rankings is to use Precision @ n ( or p@n), where n is the rank position. The MDK-LDA was measured against base model with settings p@n for n = 5, 10, 15 and 20. As a result MDK-LDA outperforms base LDA model in each test, the overall average precision score is 0.67 and 0.88 for LDA and MDK-LDA respectively, a paired t-test also conducted with the result of $p < 0.0001$ which means a statistically significance  between LDA and MDK-LDA. The study concluded Multi-Domain Prior Knowledge can be used to improve topic modeling. 
As most topic modeling studies, MDK-LDA studies data of product reviews. The data is from six domains from Amazon.com. And for this model to work, a prior domain knowledge set (or s-set) need to be generated. The process of generating the s-set involves certain level of human interaction. Our study focuses on Australia healthcare reviews which to the best of our knowledge there hasn't any topic modeling studies tried to exploring this area. We also proposed a new approach called user-driven topic modeling.

\section{Intrinsic qualitative evaluation of learned topics}
With the strong interest within computational linguistics in techniques for learning topics which capture the latent semantics of a document collection, many models have been proposed for example, LDA and MDK-LDA we mentioned above. Extrinsic evaluation has been used to demonstrate the effectiveness of the learned topic in the application domain but, no attempt has been made to perform intrinsic evaluation of the topics themselves either qualitatively or quantitatively, Newman (\cite{ref27}) introduces the novel task of topic coherence evaluation. A range of topic scoring models to the evaluation task are applied, drawing on WordNet, Wikipedia and the Google search engine and compared with human scores for a set of learned topics. The evaluation metrics:
\begin{equation}
Mean-D-Score(t) =  mean{D(w_{i}, w_{j}, ij\in 1…10, i<j)}
\end{equation}

\begin{equation}
Median-D-Score(t) =  median{D(w_{i}, w_{j}, ij\in 1…10, i<j)}
\end{equation}

Given the topic t based on the component terms $(w_{1}…w_{10}$ and word-similarity measure $D(w_{i}, w_{j})$, the total number of word-pair over 10 words is 45. The experiment collected 55,000 news articles from English Gigaword, and the collection of 12,000 books was downloaded from the Internet Archive. 200 and 400 topics are learned from topic modeling for news and books respectively. Then a total of 237 topics are select from two collection for user scoring. 9 users scored 237 topics on a 3-point scale where 3=``useful" (coherent) and 1 = ``useless"(less coherent). The inter-annotator agreement (IAA) score is treated as the ``Gold-standard", of all the topic scoring methods tested, Pointwise Mutual Information (PMI) is the most consistent performer, achieving the best or near-best results over both datasets. 
\begin{equation}
PMI(w_{i}, w_{j}) = log \frac{P(w_{i}, w{j})}{P(w_{i})P(w_{j})},
\end{equation}
Due to the consistency of PMI, we will use this method to measure topic coherence of learnt topics from Patient Opinion and also for intrinsic evaluation of user-driven topic modeling.

\section{Best Topic Word Selection for Topic Labelling}
Naturally, not all topics are equally coherent, and the higher the topic coherence, the easier the label selection task becomes. Lau (\cite{ref33}) proposed a method to select best topic word for topic labelling based on word probability model. The conditional probability used in the study:
\begin{equation}
P(w_{i}\mid w_{j} = \frac{P(w_{i}, w_{j})}{P(w_{j})}
\end{equation}
where $i\neq j$ and $P(w_{i}, w_{j})$ is the probability of observing both $w_{i}$ and $w_{j}$ in the same sliding window, and $P(w_{i}$ is the overall probability of word $w_{i}$ in the corpus. Then average of conditional probability for word $w_{i}$ is calculated by:
\begin{equation}
avg-CP1(w_{i}) = \frac{1}{9} \sum_{j}P(w_{i} \mid w_{j})
\end{equation}
for j = 1 … 1, $j \neq i$, as the study focus on top 10 topic words. The flipped situation also tested, where the most representative word may evoke (rather than be evoked by) other words in the list of ten word, the equation becomes:
\begin{equation}
avg-CP1(w_{i}) = \frac{1}{9} \sum_{j}P(w_{j} \mid w_{i})
\end{equation}
The study also tested the pointwise mutual information (PMI) approach we mentioned in previous section, the average of word $w_{i}$ over top 10 words in the topic is given by:
\begin{equation}
avg-PMI(w_{i}) = \frac{1}{9} \sum_{j}PMI(w_{j}, w_{i})
\end{equation}
The testing data used in this study is exactly same as Newman's (\cite{ref27}) we mentioned in previous section. The PMI-scores are calculated for 200 topics from NEWS and 400 topics from BOOKS. 60 topics are selected with high PMI-score, and 60 topics with low PMI-score, from both group, resulting in a total of 240 topics for human evaluation. Each topic is scored with 3-point scale by human where score 3 means a most useful or coherent topic. The words in each topic are ranked by average PMI and conditional probabilities CP1 and CP2, top 3 words from each approached are selected as best words candidates. They are evaluated against weighted scoring function:
\begin{equation}
Best-N score = \frac{\sum_{i=1}^{N}n(w_{rev_{i}})}{\sum_{i=1}^{N}n(w_{i})}
\end{equation}
where $w_{rev_{i}}$ is the $i^{th}$ term ranked by the system and $w_i$ is the i^{th} most popular term selected by annotators; $rev_{i}$ gives the index of the word $w_i$ in the annotator's list; and n(w) is the number of votes given by annotators for word w. As a result, each approach shows the strength in certain topics against base model, the performance isn't consistent across 3 approaches, however, in combination as inputs to a re-ranking model, the result is consistent and shows improvement against baseline. In our study of Patient Opinion Australia, PMI-scores for topics are used to evaluate topic coherence and it is measured by average of number of word-pairs, the reason is we produce topics with variable number of terms. To rank the words in each topic, we use Inverse Document Frequency given by:
\begin{equation}
idf(t, D) = log\frac{N}{\mid \{d\in D : t\in D\}\mid}
\end{equation}

%%%%%%%%%%%%%%%%%%%%%%%%%%%%%%%%%%%%%%%%%%%%%%%%%%%%%%%%%%%%%%%%%%%%%%
\chapter{User-driven Topic Modeling}

Although LDA provides a powerful framework for extracting latent topics in text document, sometimes the learned topics extracted are lists of words that do not convey much useful information (\cite{ref26}). To solve the problem base LDA model had been extended either by incorporating human judgment into the model-learning framework or creating a computational proxy that simulates human judgments (\cite{ref28}) as per the MDK-LDA model (\cite{ref24}) we discussed in Chapter 2. However, solutions like MDK-LDA require a certain level of human interaction which limits the size of the training data. Part of the training constraints are defined by domain experts which means the result could contain certain level of subjective factors.

Due to the unique characteristic of the data of Patient Opinion, we have the opportunity to minimize the input from human judge during training phase by employing use user input to simulate human judgment, hence to produce the topic modeling result with less subjective factors.

% ~~~~~~~~~~~~~~~~~~~~~~~~~~
\section{Description of Data}
The data from Patient Opinion of relevance to this study includes:
\begin{itemize}
\item The title of the story and the summary of the story by the user.
\item The ``Author role" and ``Time of the post", author role can be the patient, patient’s relative, carer or doctor.
\item The ``More about" section completed by the website moderator who inserts relevant tags to the story.
\item The most important fields are ``What is Good" and 
``What could be improved", these field are inserted by the user, the fields indicate what user thinks the story is about, and we use these fields to simulate user judgement in topic modeling.
\end{itemize}

\begin{figure}[h]
    \begin{center}
    \epsfig{figure=Figs/story_sample.jpg,width=16.7cm}
    \caption
    [Patient Opinion Story Sample]
    {
    Patient Opinion Story Sample
    \label{Figure1}
    }
    \end{center}
\end{figure}

\begin{figure}[h]
    \begin{center}
    \epsfig{figure=Figs/story_sample_source1,width=16.7cm}
    \caption
    [Patient Opinion Story Sample Source 1]
    {
    Patient Opinion Story Sample Source 1
    \label{Figure2}
    }
    \end{center}
\end{figure}

\begin{figure}[h]
    \begin{center}
    \epsfig{figure=Figs/story_sample_source2,width=10.7cm}
    \caption
    [Patient Opinion Story Sample Source 2]
    {
    Patient Opinion Story Sample Source 2
    \label{Figure3}
    }
    \end{center}
\end{figure}
Each story also includes some non-mandatory fields which are not in the screenshot but provide very important information for example, the location information as some stories specify a particular hospital or clinic and most stories at least have state information. A quick overview of the data illustrates the potential of using the data from a public forum for monitoring of healthcare professionals and institutions. Patient Opinion Australia established was in 2012, and to date contains 624 stories in total. It's allied site Patient Opinion UK was founded in 2005 and has more than 80,000 stories. As the scope of this project's focus is on the Australian healthcare system, the data from the United Kingdom will not be used but it is evident that the healthcare system is attracting feedback from general public. The information collected from those sources has the potential to improve and influence conduct of healthcare professionals and institutions. From Patient Opinion Australia we collected 659 unique terms in user specified field out of 624 stories. 

The most frequent terms are: ``care" which appears 399 times in 278 stories; ``service" which appears 150 times in 141 stories; ``staff" which appears 148 times in 116 stories; and ``hospital" which appears 141 times in 118 stories. The terms where split into to two groups: ``Good" and ``Need to Improve", we have 412 user specified terms in ``Good" out of 467 stories, while the ``Need to Improve" includes 408 terms in 264 stories. An interesting observation is the general order of term frequency in ``Good" group matches overall count, while ``Need to Improve" group shows some differences. Instead of ``service" and ``staff", ``hospital, doctor, communication" follows after ``care" in this group. It suggests stories relate to ``service, staff" are more likely get positive feedback compared to ``hospital, doctor, communication". However, both groupings illustrate the interest from general public as we do not want to overlook topics which positive. When data was analysed by the topic modeling algorithm all user specified terms were treated as a single group. Table \ref{table:OverAllCount} displays the term frequency per state of Australia and New South Wales and Queensland are the most active states. Table \ref{table:GoodCount} includes the ``Good" group term frequency per state. The term ``information" has been mentioned 54 times out of 49 stories with 42 stories are from New South Wales. These findings suggest patients in New South Wales are more satisfied in ``information" related topics than other states. Table \ref{table:ImproveCount} includes the result for ``Need to Improve" group term frequency per state. The term frequency of  ``care, doctor, communication, staff" are markedly higher for Queensland than the other states . 

Each number we had showed so far is individual case for example, we assumed the issue with ``service", ``staff", ``hospital" are all isolated cases, in reality an issue could be ``bad service from staff"; ``issues with staff in hospital", topic modeling is a process to discover the connections between terms. Our project intend to employ traditional topic modeling technique to discover the latent topics among documents, and then increase the accuracy of the result with the help of user specified keywords.


\begin{table}[h]
\caption{Overall Term Count Over States}
\centering
\begin{tabular}{| c | c | c | c | c | c | c | c |c | c | c |}
\hline\hline
Term & $TF_{ALL}$ & $DF_{ALL}$ & $DF_{NSW}$ & $DF_{VIC}$ & $DF_{ACT}$ & $DF_{TAS}$ & $DF_{QLD}$ & $DF_{SA}$ & $DF_{NT}$ & $DF_{WA}$ \\
\hline
care & 399 & 278 & 61 & 28 & 4 & 3 & 114 & 21 & 2 & 5 \\
\hline
service & 150 & 141 & 51 & 7 & 1 & 0 & 51 & 5 & 1 & 5 \\
\hline
staff & 148 & 116 & 32 & 12 & 1 & 0 & 45 & 4 & 0 & 4 \\
\hline
hospital & 141 & 118 & 18 & 18 & 4 & 3 & 47 & 10 & 1 & 4 \\
\hline
doctor & 102 & 84 & 16 & 7 & 2 & 1 & 40 & 6 & 1 & 0 \\
\hline
\end{tabular}
\label{table:OverAllCount}
\end{table}

\begin{table}[h]
\caption{Group ``Good" Term Count Over States}
\centering
\begin{tabular}{| c | c | c | c | c | c | c | c |c | c | c |}
\hline\hline
Term & $TF_{ALL}$ & $DF_{ALL}$ & $DF_{NSW}$ & $DF_{VIC}$ & $DF_{ACT}$ & $DF_{TAS}$ & $DF_{QLD}$ & $DF_{SA}$ & $DF_{NT}$ & $DF_{WA}$ \\
\hline
care & 250 & 178 & 48 & 13 & 2 & 2 & 73 & 11 & 2 & 6 \\
\hline
service & 125 & 119 & 47 & 5 & 1 & 0 & 42 & 4 & 1 & 5 \\
\hline
staff & 124 & 98 & 30 & 1 & 1 & 0 & 35 & 3 & 0 & 3 \\
\hline
hospital & 80 & 68 & 8 & 9 & 1 & 2 & 31 & 5 & 1 & 2 \\
\hline
information & 54 & 49 & 42 & 0 & 0 & 0 & 4 & 1 & 0 & 1 \\
\hline
doctor & 46 & 40 & 11 & 2 & 1 & 0 & 17 & 4 & 1 & 0 \\
\hline
\end{tabular}
\label{table:GoodCount}
\end{table}

\begin{table}[h]
\caption{Group ``Need to Improve" Term Count Over States}
\centering
\begin{tabular}{| c | c | c | c | c | c | c | c |c | c | c |}
\hline\hline
Term & $TF_{ALL}$ & $DF_{ALL}$ & $DF_{NSW}$ & $DF_{VIC}$ & $DF_{ACT}$ & $DF_{TAS}$ & $DF_{QLD}$ & $DF_{SA}$ & $DF_{NT}$ & $DF_{WA}$ \\
\hline
care & 149 & 115 & 16 & 15 & 2 & 1 & 47 & 12 & 1 & 3 \\
\hline
hospital & 61 & 54 & 11 & 8 & 3 & 0 & 15 & 4 & 1 & 1 \\
\hline
doctor & 56 & 44 & 5 & 5 & 1 & 1 & 23 & 2 & 0 & 0 \\
\hline
communication & 26 & 26 & 2 & 1 & 1 & 0 & 15 & 5 & 0 & 0 \\
\hline
service & 25 & 23 & 4 & 2 & 0 & 0 & 9 & 1 & 0 & 0 \\
\hline
staff & 24 & 22 & 2 & 1 & 0 & 0 & 12 & 1 & 0 & 1 \\
\hline
\end{tabular}
\label{table:ImproveCount}
\end{table}

% ~~~~~~~~~~~~~~~~~~~~~~~~~~
\section{Preprocessing of Data}

To be able to feed the data to Mallet, the original data was normalized. The story body and story identification (ID) are extracted from website. All stories are condensed into one single file with each story formatted to a single line start with the story identification. The leading story identification is used by Mallet for labelling the composition result. Some internal data structures are also defined for post processing. All data is converted to lowercase. Unique user specified keywords are collected. This collection is used to filter out the words in each topic that is generated by LDA at a later stage. A list of related document identification to each word is also collected and indexed for fast lookup for document frequency. Figure 3.4 illustrates the basic data structure: documents that contain term 'education' are '58192, 59160, 58385, 59177…', the numbers are the document identification from Patient Opinion Australia, the original identification is used for easier retrieval of stories.

\begin{figure}[tp]
    \begin{center}
    \epsfig{figure=Figs/data_sample1.jpg,width=8cm}
    \caption
    [Patient Opinion Story Sample]
    {
    Patient Opinion Story Sample
    \label{Figure4}
    }
    \end{center}
\end{figure}

% ~~~~~~~~~~~~~~~~~~~~~~~~~~
\section{Using user input to improve topic modeling result}

Topics learned from LDA may sometimes not convey much useful information. This may be caused by overfeeding the result set for example, it includes top 19 words for each topic (based on the settings, the total number in each topic can be configured) and some words may not be relevant to the current topic but statistically significant to the topic. Our goal  is using user input to reduce the noise while retaining as much information as possible for a topic. The process is given as follows: 
\begin{enumerate}
\item Collect unique words from user specified field as \textit{S} set.
\item Generate a set of topics \textit{T} with LDA model.
\item Calculate result set \textit{R} as: for each topic $t\in\{1,...,T\}, r_n = t_n \cap S$
\item Treat the result T as collection of document, calculate Inverse Document Frequency for each term with $idf(t, D) = log\frac{N}{\mid \{d\in D : t\in D\}\mid}$, where N = 100 represent the 100 topics generated with LDA. 
\item Re-arrange terms in set R with idf score, so the more significant word appear in front of each topic.
\end{enumerate}

When collect unique user specified keywords the phrase is break down to words, because the result topic generated by LDA is collection of words. Inverse Document Frequency of each word is calculated in the scope of topic, it is the measurement of how important a term is within topics. 
%%%%%%%%%%%%%%%%%%%%%%%%%%%%%%%%%%%%%%%%%%%%%%%%%%%%%%%%%%%%%%%%%%%%%%
\chapter{Experiments and Result}

We collected all 624 stories from Patient Opinion by August 2014. The count of unique user specified term is 659. One hundred topics are generated using Mallet \footnote{http://mallet.cs.umass.edu/} with setting of optimize interval equals to 20.  The number of unique terms in the overall topic set and also in user specified keywords set is 527, compare to 1440 unique terms in the overall topic set. One of the metrics Mallet provide is Topic Composition which is per topic probability distributions over current document ( $P(T \mid D)$ ) which in other word the probability of T given document D. For each document the sum of composition over total topic should always be 1:

\begin{equation}
\sum_{t=1}^100 P(t\mid D) = 1
\end{equation}

Table \ref{table:Composition} includes the topic composition. Rows represent documents with document identification (the document ID is the unique number from Patient Opinion Australia, the number can be used to retrieve the original document, not the index of document as by today there are only 624 stories in total) in first column and the remaining columns represent topic probability distributions over current document. 

As each topic can be represented as probability distribution of documents and document is formed by terms we can use the sum of composition score to measure topic quality:

\begin{equation}
Score(t) = \sum_{n=1}^N Comp(D_{n,t})
\end{equation}

Where N is set of document that contains at least one term in topic ``t", $Comp(D_{n,t})$ is the composition of topic ``t" of document ``n". 
The total composition score for each topic is calculated with the help from term to document identification index built previously (Table \ref{table:SumComposition}). Clearly, the total composition of documents from original topic words is expected to be greater than its subset which is the same topic but filtered the words those not mentioned by user at all. If the sum of topic composition is small between two groups means the topic we produce has the similar quality as the original one, in other words the documents only contain the terms we omitted ( which not mentioned by user at all) do not have significant contribution to the topic, as a result those terms are considered as noise from base model.
The average difference of sum of composition between two groups over 100 topics is 1.0829. There are 58 topics which have the difference below this average number and use this number as cut-off boundary, we consider those 58 topics we extracted from base model are good topics as the quality of the topic are relatively higher than other topics. A sample of the topics are selected from the remaining topics. Please refer to Table \ref{table:SampleTopics}.
\begin{table}[h]
\tiny
\caption{Example of Composition}
\centering
\begin{tabular}{| c | c | c | c | c | c | c | c | c | c |}
\hline\hline
Doc ID & Topic Index & Composition & Topic Index & Composition & ... & Topic Index & Composition & Topic Index & Composition\\
\hline
58954 & 61 & 0.1171875 & 91 & 0.0390625 &...& 78 & 0.0234375 & 72 & 0.0234375\\
\hline
58832 & 83 & 0.076388889 & 78 & 0.048611111 &...& 10 & 0.048611111 & 71 & 0.034722222\\
\hline
58953 & 83 & 0.077380952 & 65 & 0.053571429 &...& 29 & 0.041666667 & 60 & 0.029761905\\
\hline
58956 & 78 & 0.025 & 76 & 0.025 &...& 65 & 0.025 & 62 & 0.025\\
\hline
58834 & 42 & 0.065517241 & 11 & 0.065517241 &...& 58 & 0.037931034 & 44 & 0.037931034\\
\hline
58710 & 12 & 0.108208955 & 18 & 0.063432836 &...& 90 & 0.041044776 & 71 & 0.041044776\\
\hline
58952 & 91 & 0.044642857 & 73 & 0.026785714 &...& 59 & 0.026785714 & 36 & 0.026785714\\
\hline
58830 & 94 & 0.108490566 & 80 & 0.051886792 &...& 99 & 0.04245283 & 71 & 0.04245283\\
\hline
58951 & 93 & 0.0625 & 81 & 0.052884615 &...& 0 & 0.052884615 & 79 & 0.043269231\\
\hline
58719 & 51 & 0.27238806 & 27 & 0.063432836 &...& 42 & 0.026119403 & 22 & 0.026119403\\
\hline
58716 & 77 & 0.041666667 & 90 & 0.025 &...& 73 & 0.025 & 62 & 0.025\\
\hline
58718 & 83 & 0.242307692 & 47 & 0.05 &...& 71 & 0.042307692 & 11 & 0.034615385\\
\hline
58839 & 25 & 0.070512821 & 63 & 0.032051282 &...& 59 & 0.032051282 & 58 & 0.032051282\\
\hline
58717 & 43 & 0.050660793 & 57 & 0.046255507 &...& 92 & 0.04185022 & 72 & 0.04185022\\
\hline
58723 & 91 & 0.028301887 & 68 & 0.028301887 &...& 35 & 0.028301887 & 99 & 0.009433962\\
\hline
58965 & 83 & 0.097014925 & 1 & 0.037313433 &...& 77 & 0.02238806 & 72 & 0.02238806\\
\hline
\end{tabular}
\label{table:Composition}
\end{table}

\begin{table}[h]
\caption{Sum of Composition}
\centering
\begin{tabular}{c c c}
\hline\hline
Topic Index & Sum of Composition of Original Documents & Sum of Composition of Filtered Documents\\
\hline
1 & 3.920362 & 3.382284\\
2 & 3.742091 & 3.275150\\
3 & 4.396231 & 3.930039\\
4 & 4.275353 & 3.603821\\
5 & 4.569670 & 4.444751\\
6 & 3.153133 & 2.775077\\
7 & 3.368214 & 2.622281\\
8 & 4.541646 & 4.018145\\
9 & 5.399709 & 5.194911\\
10 & 4.498450 & 3.711763\\
11 & 4.255079 & 4.101976\\
12 & 6.755018 & 6.489172\\
\hline
\end{tabular}
\label{table:SumComposition}
\end{table}

Inverse Document Frequency (idf) is used to measure the weight of each term in the scope of whole topic set which means the less the term appears in overall topic set the more important the term is. One thousand one hundred and sixteen out of original 1440 terms appears once in original topic set, so more than half of the term has the highest idf value 2 in the data set. Two hundred and thirty six terms appear twice with idf=1.7 and 55 terms appear 3 times with idf = 1.5. There are also significant variance in the result topics as some topics may remain with the existing order or only have one word shifted, whilst some may look very differently. For example, \{program, kate, lifestle, meet, included, learnt, learning, relationship\}, only $idf_{kate} = 1.7$, the idf for the rest equals to 2, so the new topic becomes \{program, kate, meet, included, learnt, learning, relationship\}. Another example \{bed, family, care, staff, time, attending, unit, palliative\}, $idf_{care}=1.3, idf_{staff}=1.2 and idf_{palliative}=2$ and the rest has idf equals to 1.7, as a result the new topic should look like \{palliative, bed, family, time, attending, unit, care, staff\}. 

\begin{table}[h]
\tiny
\caption{Sample topics}
\centering
\begin{tabular}{|c|c|}
\hline\hline
Original & Filtered\\
\hline
\makecell{program healthy kate lifestyle sessions eat meet programme\\
	 included organised held healthier encouraging learnt foods\\
	 beneficial learning relationship handle} & {program kate lifestyle meet included learnt learning relationship} \\
\hline

\makecell{gp local recently records government copy prescription\\
	 paper multiple tasmania gps referring avail beginning calls\\
	  surprised cairns super shared } & {gp local records government prescription paper gps cairns}\\
\hline

\makecell{physio gp mri injury follow shoulder xray week asked\\
	 hospital discussed full physiotherapist neck stand \\
	 complaining neurologist princess forte} & {physio gp mri xray hospital full physiotherapist neck}\\
\hline

\makecell{call waiting phone told back called list unit\\
	 rang ring explain apparently assumed clerk calling\\
	 noticed mcewin lyell requested} & {call waiting phone back list unit clerk calling}\\
\hline
  
\makecell{father bed family care appears staff time\\
	 attending difficult dad speak comfort unit incident\\
	 law visitor awake gosford palliative} & {bed family care staff time attending unit palliative}\\
\hline
	
\makecell{time advised team contact causing consultant tumour\\
	 professional manner independent safe stressful arrival\\
	 note closed usual considerate empathetic seizures} & \makecell{time team contact consultant professional manner\\ independent empathetic}\\
\hline

\makecell{looked experience er bad partner full approach worry\\
	 give free skills male chronic terrible running provider\\
	  building drive welcomed} & {looked experience er partner full approach skills male building}\\
\hline

\makecell{night stay thing major hospital admission support\\
	 suggested accommodation fully sydney unable relatives\\
	 sleep developed tuesday staff added environment} & \makecell{night stay hospital admission accommodation relatives\\ sleep staff environment}\\
\hline

\makecell{waiting wait hours room hour area waited reception\\
	 number temperature hurt remember panadol minutes geelong\\
	 sunday impressed time er} & {waiting wait hours room area reception number time er}\\
\hline

\makecell{child issues jean aboriginal helped understanding\\ 
	knowledge school minds behaviour hay mighty woods anger\\ 
	louise clinician strategies interaction love} & \makecell{child jean aboriginal understanding knowledge\\ 
	school minds behaviour woods strategies}\\
\hline

\makecell{public brisbane system live mater hospital pa\\ 
	run booking advise toowoomba meds qld health expect\\ 
	west lift weekend weak} & \makecell{public brisbane system hospital run booking meds\\ health west lift}\\
\hline

\makecell{health community sarah local support primary\\ 
	medicare rural group libby people murrumbidgee provide\\ 
	part clients topics art groups guest} & \makecell{health community sarah local primary medicare rural\\ group people murrumbidgee clients guest}\\
\hline
\end{tabular}
\label{table:SampleTopics}
\end{table}

% ~~~~~~~~~~~~~~~~~~~~~~~~~~
\section{Topic Coherence Evaluation}
Apart from quantitative and qualitative evaluation as above evaluating topic coherence is a component of the larger question of what are good topics, what characteristics of a document collection make it more amenable to topic modeling and how can the potential of topic modeling be harnessed for human consumption (\cite{ref27}). The topic coherence is measured with Pointwise Mutual Information (\cite{ref32}) (PMI) score:
\begin{equation}
PMI-Score(w) = (\frac{N^{2}-N}{2})^{-1}\sum PMI(w_{i}, w_{j}), ij\in \{1...N\}
\end{equation}
where
\begin{equation}
PMI(w_{i}, w_{j}) = log \frac{P(w_{i}, w{j})}{P(w_{i})P(w_{j})},
\end{equation}
Since the number of terms that form each topic isn't normalised we calculate the average of the topic where N is the number of terms in that topic. $(\frac{N^{2}-N}{2})^{-1}$ gives the number of distinct pairs in N.
The measure is symmetric $P(w_{i}, w{j}) = P(w_{j}, w{i})$ which means we only measure the difference of topic coherence between original topic and filtered topic.

Table \ref{table:PMIScores} includes some samples of the PMI scores plus the difference between two PMI included in the last column. As we can see from the result the majority of the topics result in improvement of coherence and the overall average is 0.297. A paired-t test shows the $P=3.76e^{-22}$ which means the filtered topic is statistically significant from the original one.

\begin{table}[h]
\caption{PMI Scores}
\centering
\begin{tabular}{c c c c}
\hline\hline
Topic Index & Original Topic PMI Scores & Filtered Topic PMI Scores & Difference\\
\hline
1 & 1.864771 & 2.040466 & 0.175695\\
2 & 1.844573 & 2.305665 & 0.461092\\
3 & 1.543335 & 1.928804 & 0.385469\\
4 & 2.145729 & 2.333656 & 0.187927\\
5 & 1.912059 & 2.195632 & 0.283573\\
6 & 1.854318 & 2.341428 & 0.48711\\
7 & 1.748795 & 2.33346 & 0.584665\\
8 & 1.899982 & 2.624272 & 0.72429\\
9 & 1.752184 & 1.80262 & 0.050436\\
10 & 2.152952 & 2.371954 & 0.219002\\
11 & 1.986177 & 2.120463 & 0.134286\\
12 & 1.776924 & 2.030256 & 0.253332\\
\hline
\end{tabular}
\label{table:PMIScores}
\end{table}

\begin{figure}[h]
    \begin{center}
    \epsfig{figure=Figs/pmi.jpg, width=10cm}
    \caption
    [PMI-Score]
    {
    PMI-Score
    \label{Figure5}
    }
    \end{center}
\end{figure}

%%%%%%%%%%%%%%%%%%%%%%%%%%%%%%%%%%%%%%%%%%%%%%%%%%%%%%%%%%%%%%%%%%%%%%
\chapter{Conclusion and Future Work}
This study shows the possibility of using user input to improve topic coherence in topic modeling of healthcare related blogs. To the best of our knowledge this has not been done before. We proposed a method that use user input words as a filter for the result from LDA model. We successfully reduced the number terms in each topic while still keep the topic meaningful. The terms that have been omitted can be considered as noise which means the documents they associate to do not significantly contribute to the possibility distribution over the topic as this is evaluated by the total composition score. Hence, the overall topic coherence is improved as less noise in the topic. Furthermore, we investigated using Inverse Document Frequency to re-rank the terms in each topic. Unfortunately PMI-score evaluation is symmetric which means the order of each term in topic is not considered. An existing statistical model was not found for the evaluation. 
Our method of ranking the terms by idf isn't ideal but since the term frequency for each term for in a topic always equals to 1, this approach is reasonable for step one. Future work may utilise this approach and could be expended to count term frequency in the scope of whole collection of documents and idf still from the topic list. If the scope of the factor expended to the whole collection, it is also reasonable to rank the topics not only the terms in each topic, hence the model could make suggestions of which topic is more likely an important one.
%%%%%%%%%%%%%%%%%%%%%%%%%%%%%%%%%%%%%%%%%%%%%%%%%%%%%%%%%%%%%%%%%%%%%%

\bibliographystyle{abbrvnat}
\bibliography{Bib/strings,Bib/main}
\end{document}
\end{document}
