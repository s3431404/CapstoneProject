% RMIT University School of CS&IT
% Minor thesis template
% S.M.M. (Saied) Tahaghoghi, 2004
\documentclass[11pt,twoside]{report}
\usepackage{a4wide,caption,epsfig,fancyheadings,natbib,url}

% Place the correct values here
%Set to the original submission date when submitted amended thesis
\newcommand{\SubmissionDate}{\today}
\newcommand{\student}{Bin Lu}
\newcommand{\supervisor}{Dr Jenny Zhang, Dr Daryl D'Souza, Dr Amanda Kimpton}
\newcommand{\topic}{Topic Modelling of Patient Opinion}
\newcommand{\school}{School of Computer Science and Information Technology}
\newcommand{\program}{Masters of Computer Science}
\newcommand{\institution}{Royal Melbourne Institute of Technology}

% Use the remark command to highlight text for discussion
\newcommand{\remark}[1]{{\bf \em [\marginpar{$\Leftarrow$}#1]}}

\renewcommand{\leftmark}{\student}
\renewcommand{\rightmark}{\topic}
%\setlength{\headrulewidth}{0pt}
\setlength{\parindent}{0pt}
\setlength{\parskip}{1.5ex plus 0.3ex}

% This is the line spacing - set to 2 for draft submission to
% supervisor, 1.3 for the final submission
\renewcommand{\baselinestretch}{2.00}

\renewcommand{\captionfont}{\it}
\raggedbottom

%For Natbib Author, year citation format
% - the opening bracket symbol, default = (
% - the closing bracket symbol, default = )
% - the punctuation between multiple citations, default = ;
% - the letter n for numerical style, or s for numerical superscript
%   style, any other letter for author year, default = author year;
% - the punctuation that comes between the author names and the year
% - the punctuation that comes between years or numbers when common author lists are suppressed (default = ,);
\bibpunct{[}{]}{;}{a}{,}{;}


\begin{document}

%%%%%%%%%%%%%%%%%%%%%%%%%%%%%%%%%%%%%%%%%%%%%%%%%%%%%%%%%%%%%%%%%%%%%%
\title{{\Large\bf \topic}}
\author{
A minor thesis submitted in partial fulfilment of the requirements for the degree of
\\\program\\*[10mm]
%\epsfig{figure=Figs/rmit-coa.epsf,width=5cm}
\\\student
\\\school
\\Science, Engineering, and Technology Portfolio,
\\\institution
\\Melbourne, Victoria, Australia
}
\maketitle
\thispagestyle{empty}


%%%%%%%%%%%%%%%%%%%%%%%%%%%%%%%%%%%%%%%%%%%%%%%%%%%%%%%%%%%%%%%%%%%%%%
\chapter*{Declaration}

This thesis contains work that has not been submitted previously, in
whole or in part, for any other academic award and is solely my
original research, except where acknowledged.

This work has been carried out since TODO:MONTH TODO:YEAR, under the
supervision of {\supervisor}.

\paragraph{}
\vspace{5cm}\noindent \\\student \\
\school\\
\institution\\
\SubmissionDate

\pagenumbering{roman}

%%%%%%%%%%%%%%%%%%%%%%%%%%%%%%%%%%%%%%%%%%%%%%%%%%%%%%%%%%%%%%%%%%%%%%
\chapter*{Acknowledgements}

TODO:THANKS!

\tableofcontents
\listoffigures
\listoftables

\pagenumbering{arabic}
%%%%%%%%%%%%%%%%%%%%%%%%%%%%%%%%%%%%%%%%%%%%%%%%%%%%%%%%%%%%%%%%%%%%%%
\chapter*{Abstract}

%%%%%%%%%%%%%%%%%%%%%%%%%%%%%%%%%%%%%%%%%%%%%%%%%%%%%%%%%%%%%%%%%%%%%%
\chapter{Introduction}



Publicly available opinions and service feedback provide valuable informations for decision making for both service providers and consumers. With the help of websites, blogs, forums and social networks, it is never been so easy to express opinions and leave feedback. Analyzing the opinions becomes a challenge, not just because of the quantity of the data, most opinion from general users are free form text. The massive quantity of the data won’t be effectively used until there is a systematically approach of analyzing and summarizing. Many techniques have been proposed to solve this problem. MDK-LDA model proposed by Chen(\cite{ref24}) , the method extends the Latent Dirichlet Allocation(\cite{ref25}), the later one becoming the standard method in topic modelling and been extended in variety ways. The basic idea of LDA is treat each document in a collection as a vector of word count, each document is represented as a probability distribution over a number of topics, while each topic is represented as a probability distribution over a number of words. MDK-LDA introduces a new latent variable s in LDA to model s-sets. Each document is an admixture of latent topics while each topic is a probability distribution over s-sets. Another approach is Aspect-based Summarization(\cite{ref11}), it is usually composed of three main tasks: aspect identification, sentiment classification, and aspect rating. Generally this model is used to analyzing product review, it is designed to effectively retrieve features and sentiment for products.

Most previous studies focus on analyzing product reviews. We are interested to discover some model that suite service reviews. More specifically, reviews relate to healthcare. Study shows the effective governance is increasingly recognized as pivotal to improvements in healthcare quality(\cite{ref6}), moreover current issue of effectiveness of the authority is affected by insufficient resource and inadequate information received(\cite{ref5}).
The object we are going to study is www.patientopinion.org.au, it is a publicly available healthcare forum. It allows user to post their own healthcare related story, the stories are not restricted from patient, it can also from hospital workers, nurses or doctors. The story can be positive or negative or a bit from both side. Although the story body is free form text, user still has to follow a certain template while submit the story.

\begin{figure}[tp]
    \begin{center}
    \epsfig{figure=Figs/story_sample.jpg,width=16.7cm}
    \caption
    [Patient Opinion Story Sample]
    {
    Patient Opinion Story Sample
    \label{fig-story_sample}
    }
    \end{center}
\end{figure}

\begin{figure}[h]
    \begin{center}
    \epsfig{figure=Figs/story_sample_source1,width=16.7cm}
    \caption
    [Patient Opinion Story Sample Source 1]
    {
    Patient Opinion Story Sample Source 1
    \label{fig-story_sample_source1}
    }
    \end{center}
\end{figure}

\begin{figure}[h]
    \begin{center}
    \epsfig{figure=Figs/story_sample_source2,width=10.7cm}
    \caption
    [Patient Opinion Story Sample Source 2]
    {
    Patient Opinion Story Sample Source 2
    \label{fig-story_sample_source2}
    }
    \end{center}
\end{figure}

Due to the unique characteristic of the data from ‘Patient Opinion’, the existing models of topic modelling may not give the best result, on other hand LDA has been approved a very effective  model, and been used as a based model in many topic modelling studies. We choose LDA as our base model, and incorporate unique feature in ‘Patient Opinion’, specifically the section of ‘What’s Good’ and ‘What could be improved’. These two sections are filled in by user while submitting the story, the template is provided by the website. Generally this will be the main topic or features user want to give feedback about in the story. And we assume user labeled story 100\% accurate. The question we aim to answer in this thesis:
\begin{itemize}
\item How to use user specified features to improve the performance and accuracy in topic modelling.
\item What is the distribution of topics over locations (State level).
\end{itemize}

%%%%%%%%%%%%%%%%%%%%%%%%%%%%%%%%%%%%%%%%%%%%%%%%%%%%%%%%%%%%%%%%%%%%%%
\chapter{Related Works}

% ~~~~~~~~~~~~~~~~~~~~~~~~~~
\section{LDA}

% ~~~~~~~~~~~~~~~~~~~~~~~~~~
\section{MDK-LDA}

%%%%%%%%%%%%%%%%%%%%%%%%%%%%%%%%%%%%%%%%%%%%%%%%%%%%%%%%%%%%%%%%%%%%%%
\chapter{The Approach}

% ~~~~~~~~~~~~~~~~~~~~~~~~~~
\section{A Section}

% ~~~~~~~~~~~~~~~~~~~~~~~~~~
\section{Another Section}

%%%%%%%%%%%%%%%%%%%%%%%%%%%%%%%%%%%%%%%%%%%%%%%%%%%%%%%%%%%%%%%%%%%%%%
\chapter{Experiments}

% ~~~~~~~~~~~~~~~~~~~~~~~~~~
\section{A Section}

% ~~~~~~~~~~~~~~~~~~~~~~~~~~
\section{Another Section}

%%%%%%%%%%%%%%%%%%%%%%%%%%%%%%%%%%%%%%%%%%%%%%%%%%%%%%%%%%%%%%%%%%%%%%
\appendix
\chapter{Testbed Configuration}

\bibliographystyle{abbrvnat}
\bibliography{Bib/strings,Bib/main}
\end{document}
\end{document}
