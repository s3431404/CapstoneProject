\documentclass[a4paper,11pt]{article}
\usepackage{graphicx}
\title{Analyzing complaint data for healthcare}
\author{Bin Lu, S3360014\\Supervisor: Xiuzhen Jenny Zhang, Daryl D'Souza, Amanda Kimpton\\Program: Minor thesis}

\begin{document}
\maketitle
\section{Introduction}
The provision of high-quality healthcare service is an increasingly difficult challenge, on the other hand the effective governance is increasingly recognized as pivotal to improvements in healthcare quality[ref]. The Australian Health Practitioner Regulation Agency (AHPRA) plays a crucial role in this aspect; it works with 14 National Health Practitioner Boards in implementing the National Registration and Accreditation Scheme. AHPRA also accept consumers or patients’ complaints, the AHPRA categorize the complaints based on the nature of the practice and then distribute them to the corresponding board. The board members will review the complaints in regular meeting to identify the issues and make suggestions; set new standards to improve the quality of service. Previous study[ref] shows the effectiveness of the boards is effected by insufficient resource and inadequate information received. Manually processing and analyzing this type of data is time consuming and inefficient. Moreover, even highly professional people could put subjective thinking into analyzes. Modern data mining technology already been employed in many area. For example market database system will analyze customers, categorize them in different groups and forecast their behavior. There is a huge potential to introduce data mining system into healthcare service. Not only reducing the labor required to process the massive amount of data, also tend to produce more accurate and objective analysis. Forecast the possibility of complaint against a medical practitioner based on historical data is a very valuable indicator for board members when they making the decision.
Although data mining techniques for marketing, manufacturing, security etc. is ripe for deployment, to apply data mining techniques to healthcare data is not a simple matter of switching content due to the complexity and sensitivity of the data. It is not a single task to develop one data mining package to cover all 14 boards, the variables and attributes affecting the result are different from board to board. There won’t be sufficient time and resource to develop algorithms to cover them all, however we are trying to find a generic data mining package to analyze the common variables among 14 boards, and develop an in depth analyze against one or two specialized boards, the similar method can be adapt to other boards with minimum amount of work in the future.

\section{Related Work}
\subsection{A statistical study of healthcare complaint data}
Unfortunately, data miming against healthcare complaint hasn’t attracted much attention,  some studies has been done recently to discover the relationship between complaint data against particular or particular group ( gender, age, specialty ) of doctors with one decade worth of official complaint data within healthcare system. The study analyzed the distribution of complaints among practicing doctors. Then used recurrent-event survival analysis to identify characteristics of doctor at high risk of recurrent complaints, and to estimate each individual doctor’s risk of incurring future complaints. The study shows some very interesting and inspirational ideas, it coded specialty into 13 categories, for example: General Practice; Surgery; Psychiatry etc. The issues or complaints also categorized into 20 sections, for example Medication, Treatment in Clinical care; Consent, Information in Communication; Cost, Billing, Sexual Contact in Conduct etc. The result then been project to groups of doctors based on their gender, age and location (Urban/Rural). 

\section{Project Plan}

\section{Evaluation}

\end{document}
